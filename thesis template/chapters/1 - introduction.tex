\chapter{Introduction to the Ballbot as a Dynamical System}

The ballbot is a recently patented (REF CITE PATENT) robot with a unique 
structure. That structure consists of a spherical base combined with a tall, 
narrow body. Unlike other, statically stable robots, the ballbot is inherently
unstable, however the narrow and vertical body make it of significant interest
for human-interactive robotics and navigation within flat environments.

Physically, the ballbot consists of two moving and interlinked parts: the ball
and the body. The ball is a non-massless sphere which remains in constant
contact with the ground. The body, which houses the control hardware and power
supply, balances on top the ball. In practical application, there are typically
three points of contact, each of which is a attached to a motor which drive the
wheel and allow the robot to balance and move

From a control theory perspective, one can see that a ballbot isn't derived from
any other system, but is instead, related to a 3D inverted pendulum. More
specifically, the ball is modeled as a uniformly dense sphere which is attached
to a uniformly dense cylinder. This system is inherently unstable under open-loop conditions, only having one
unstable equilibrium when the ballbot is perfectly vertically aligned. Without
active control, the ballbot will rapidly diverge from this equilibrium.

The ballbot currently has limited commercial uses, it's distinct structure,
ability to maneuver narrow spaces, and ability for zero-turn rotation make it of
interest for modern development in fields ranging from logistics, to security,
or even personal robotics.

As will be described in subsequent chapters, we define the ballbot as having 2
perpendicular inputs representing torques on the ball, and 8 states,
representing positions, angles, velocities and angular velocities of both the
ball and the body. We will not be modeling the yaw rotation of the body, nor
will we assume that slipping can occur between the ball and the ground. Finally,
the rotation of the ball itself will not be modeled as that would transform
the system into a non-holonomic system for no practical gain.