\chapter{Introduction to the Unicycle as a Dynamical System}

In the modern day, the unicycle is thought of as having the form as shown in 
Figure \ref{fig:patent_unicycle}. This form was patented in 1963 and serves
as the most common example of a unicycle

This form of a unicycle consists of two moving and interlinked parts: the fork
and the wheel. The wheel stays in constant contact with the ground and is
connected to the hub in it's center. This hub, in turn is connected to the
lowest part of the fork. On the other end of the fork, there is a seat for the
user.

From inspection, it is possible to see that this has a form similar to that
of a cart with an attached inverted pendulum. For the purposes of this document,
we will assume that 

As will be described in subsequent chapters, we define the unicycle as having 
three inputs in the form of torques and 5 outputs in the form of positions and
angular rotations.