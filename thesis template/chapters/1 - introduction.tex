\chapter{Introduction to the Holonomic Ballbot as a Dynamical System}

The ballbot is a recently patented \cite{ballbot_patent} robot with a unique 
structure consisting of a spherical base combined with a tall, narrow, body as
shown in Figure \ref{fig:ballbot_patent}. Unlike other, statically stable robots
, the ballbot is inherently unstable. However,its unique shape make it of 
significant interest for modern robotics and agile navigation within flat
environments, despite its currently limited uses.

Physically, the ballbot consists of two moving and interlinked parts: the ball
and the body. The ball is a sphere which remains in constant contact with the 
ground while the body is an elongated cylinder which houses the control hardware
, power supply, and balances on top the ball. In practical application, there
are typically three points of contact between the base and the ball, each of which
is attached to a motor and allow the body to drive the wheel and, in turn, allow
the robot to balance and move.

From a controls perspective, the ballbot is related to a 3D inverted pendulum.
More specifically, the ball is modeled as a uniformly dense sphere which is
attached to a uniformly dense cylinder. This system is inherently unstable under
open-loop conditions, having one unstable equilibrium when the ballbot is
perfectly vertically aligned. Without active control, in a non-idealized
environment, the ballbot will rapidly diverge from this equilibrium and fall.

As will be described in subsequent chapters, we define the ballbot as having two
inputs representing perpendicular torques on the ball alongside 8 states,
representing positions, angles, velocities and angular velocities of both the
ball, and relation to it. Constructing these two torques from the practical
three motor input is possible, but will not be discussed here. We will also
assume an environment without slipping between the ground and the ball.
Finally, we will not be modeling the yaw rotation of the body nor the
rotation of the ball itself, as both create a non-holonomic system rendering it
unable to be stabilized using static state feedback
\cite{nonholonomic_no_static_state_feedback} as discussed in class.