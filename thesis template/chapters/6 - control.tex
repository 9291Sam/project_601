\chapter{Control}

The goal for this project was to create a demonstration of the ballbot moving
in a figure eight pattern. More specifically, this means, starting the initial 
$\state$ at some arbitrary position and angle and with an observer initialized
to $\hat{\state} = 0$ Since the observer has already been
defined, we still need to define the controller and the reference function for 
the figure eight's path.

The path traced by a figure eight is nontrivial and requires a balance between
position and angle. If the controller focuses too much on the position, the
robot will move too quickly and fall over. On the other hand, if the controller
focuses too much on angle, it won't fall over, but the path traced will be
incorrect.

\section{Definition of figure eight reference function}

A reference signal is function that evaluates to a desired state of the system.
First, recall that the state vector is
$\state = \bmat{x&y&\theta&\phi&\dot{x}&\dot{y}&\dot{\theta}&\dot{\phi}}^\top$. 
Second, we can see, upon inspection that the only states which matter in a 
figure eight pattern are the $x$ and $y$ states of the base of the robot.

In order to derive the equation for this path, we start with the parametric form
of a circle of radius $r$ as $(x, y) = (r \cos(\omega t), r\sin(\omega t))$. From this form, one
notices that the parametric form of a figure eight is related to the parametric
form of a circle where one component moves at twice the speed over half the
distance. This can be stated mathematically as 
$$(x, y) = (r \cos(\omega t), 0.5 \cdot r\sin(2\omega t)).$$

Finally, this can be written in the form of a desired state vector with respect
to time ($t$)
$$\state(t) = \bmat{
    r \cos(\omega t) \\
    0.5 r \sin(2 \omega t) \\
    0 \\ 
    0 \\
    0 \\ 
    0 \\
    0 \\ 
    0 \\
}.$$

\section{Controller}

As before, when designing the observer, we will use LQR to place our poles.
However, unlike before, we will put a stronger emphasis on the position as, 
we don't partially care what the body angles are, only that they are not
extreme enough to cause the ballbot to fall over.

By trial and error, the gains we selected were
$Q = 10 \cdot \text{diag}(10, 10, 1, 1, 1, 1, 1, 1)$ and $R = I_{2x2}$
as they appeared to strike a nice balance between acceptable tracking of the 
reference vector without the robot falling over.


By using the built-in matlab function, I got a K matrix of
$$K = \bmat{
    0 & -3.1623 & 0 & 32.1665 & 0 & -4.2961 & 0 & 6.9921 \\
    3.1623 & 0 & 32.1665 & 0 & 4.2961 & 0 & 6.9921 & 0
}$$

This is used in static state feedback with the system with the input vector $u$
being calculated as 
$$u = -K(\hat{\state} - r)$$

\section{Results}

With all of that implemented, below in Figures 
we demonstrate a figure eight with radius $r = 3.0$ and a period $\omega = 2\pi/5$

\begin{figure}[h!]
  \centering
  \includegraphics[width=0.99\linewidth]{final_controlled_x.png}
  \caption{Graph of the $x$ coordinate while a figure eight pattern is traced}
  \label{fig:final_controlled_x}
\end{figure}


\begin{figure}[h!]
  \centering
  \includegraphics[width=0.99\linewidth]{final_controlled_y.png}
  \caption{Graph of the $y$ coordinate while a figure eight pattern is traced}
  \label{fig:final_controlled_y}
\end{figure}


\begin{figure}[h!]
  \centering
  \includegraphics[width=0.99\linewidth]{final_controlled_theta.png}
  \caption{Graph of the $\theta$ coordinate while a figure eight pattern is traced}
  \label{fig:final_controlled_theta}
\end{figure}


\begin{figure}[h!]
  \centering
  \includegraphics[width=0.99\linewidth]{final_controlled_phi.png}
  \caption{Graph of the $\phi$ coordinate while a figure eight pattern is traced}
  \label{fig:final_controlled_phi}
\end{figure}


\begin{figure}[h!]
  \centering
  \includegraphics[width=0.99\linewidth]{final_controlled_dotx.png}
  \caption{Graph of the $\dot{x}$ coordinate while a figure eight pattern is traced}
  \label{fig:final_controlled_dotx}
\end{figure}


\begin{figure}[h!]
  \centering
  \includegraphics[width=0.99\linewidth]{final_controlled_doty.png}
  \caption{Graph of the $\dot{y}$ coordinate while a figure eight pattern is traced}
  \label{fig:final_controlled_doty}
\end{figure}


\begin{figure}[h!]
  \centering
  \includegraphics[width=0.99\linewidth]{final_controlled_dottheta.png}
  \caption{Graph of the $\dot{\theta}$ coordinate while a figure eight pattern is traced}
  \label{fig:final_controlled_dottheta}
\end{figure}


\begin{figure}[h!]
  \centering
  \includegraphics[width=0.99\linewidth]{final_controlled_dotphi.png}
  \caption{Graph of the $\dot{\phi}$ coordinate while a figure eight pattern is traced}
  \label{fig:final_controlled_dotphi}
\end{figure}


\chapter{Conclusion}

In conclusion, even thought I was forced to change my system at basically the 
last possible moment, I still feel that I was able to put together a project
which works and I understand. 

Before this project, I had always had a general distaste for Matlab as a
programming language, however I actually understand its purpose now. I was
previously evaluating it as a programming language when, in reality, it's 
exactly what it says in the name, it's a place where you can type math as it
appears in a math class and it will work. 

Frankly, I enjoyed doing this a lot more than I thought I was going to and,
even though I spent most of the time banging my head against the wall, I feel
accomplished with what I have done.

With that being said, I do think you should remove the unicycle from the list of
suggested systems as it is just unfair. Maybe I was modeling it and approaching
the problem wrong, but given that I was able to switch to the ballbot system and
actually succeed in an extremely compressed time frame, I think the system is
simply too complex to be accurately modeled and understood after this
introductory class.