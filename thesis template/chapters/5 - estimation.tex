\chapter{Estimation}

For feedback control to be implemented, one needs to have a measurement of the
entire state, $\state$, of a system. When working with a purely mathematical
simulation, this is trivial as one can simply access the desired state and use
it. However, in practical application, the full state is often not directly
available. Sensors can only sense some states, therefore some type of estimation
is needed to model the impossible to measure states in terms of the possible to
measure ones.

The observability of a system is wholly defined as a function of the $C$ matrix
and this can be easily seen from imagining an empty $C$ matrix and an identity
$C$ matrix with the same size as $A$. With this in mind, the number of
observable states can be determined by the rank of the observability matrix
$$\mathcal{O} = \begin{bmatrix}
    C \\
    CA \\
    CA^2 \\
    \vdots \\
    CA^{n-1}
\end{bmatrix}.$$


\section{Observability of the ballbot}
In the context of a ballbot without external tracking, there is no sensor that
can measure $x$, $y$ without an external reference. $\dot{x}$ or $\dot{y}$ can
be approximated via the use of optical sensors such as those in the bottom of a 
computer mouse. Finally, $\theta$, $\phi$, $\dot{\theta}$ and $\dot{\phi}$ are
the most measurable states because they can be measured with a gyroscope. One
should note here that $\ddot{x}$ and $\ddot{y}$ can be measured by the use of an
accelerometer, but those are not in the state and therefore not helpful to us
as we are not in the realm of practical application.

With this in mind, we will look at some choices of outputs and see what they
imply with regards to the observability of the system as a whole.

\paragraph{Case 1: Position states $x$ and $y$} 

Defining $C = \bmat{1&0&0&0&0&0&0&0\\0&1&0&0&0&0&0&0}$ we can solve for 
$rank(\mathcal{O})$ and find it to be 8, implying that the system is fully
observable. Despite this theoretical result, we were unable to get a simulation
working with this restricted set of outputs.

\paragraph{Case 2: Gyroscope measureable states $\theta$, $\phi$,
$\dot{\theta}$ and $\dot{\phi}$} 

By following the same method as above, defining 
$C = \bmat{0&0&1&0&0&0&0&0\\0&0&0&1&0&0&0&0\\0&0&0&0&0&0&1&0\\0&0&0&0&0&0&0&1}$
we can find the $rank(\mathcal{O})$ and determine it to be 6. Upon inspection 
of the resultant matrix, one will find that only the $x$ and $y$ position
states to be not observable. While this seems like a useless result at first,
this combination of outputs (sensors) and observable states makes it the perfect
$C$ matrix if we were creating a robot that needed to self balance.

\paragraph{Case 3: Hybrid states $x$, $y$, $\dot{\theta}$ and $\dot{\phi}$}

For the purposes of this document, we have chosen the output matrix to be
$C = \bmat{1&0&0&0&0&0&0&0\\0&1&0&0&0&0&0&0\\0&0&0&0&0&0&1&0\\0&0&0&0&0&0&0&1}$.
This choice creates a balance where the system is both fully observable, is actually
able to be simulated, and makes the observer fulfil a purpose as there are some 
states whose values can only be estimated through the use of the observer. Since
we will need to know the precise x and y positions for our desired simulation,
we cannot get around the fact that we need those states to be visible in the
output. It should be stated that this choice of $C$ has no specific physical
interpretation, this is just a nice middle ground set of outputs that is neither
trivial nor extremely difficult.

\pagebreak
\section{Design of the Observer}

For this document, we will use a standard Luenberger Observer taking the form
\begin{align*}
    \dot{\hat{\state}} &= f(\hat{\state}, u) - L(y - \hat{y}) \\
    \hat{y} &= C\hat{\state}
\end{align*}
where $L$ is the observer's gain. To select the observer's gain, we will use
matlab's lqr function, passing in the linearization $A^\top$ and $C^\top$.
Selecting the gains was done by trial and error and eventually resulted in
state gains of $Q = 10 \cdot \text{diag}(1, 1, 25, 25, 1, 1, 5, 5)$ and input
gains of $R = I_{4x4}$ which we found to work well for our application.

After execution and transposing to get our proper form, the resultant gain
matrix is
$$
L = \bmat{
    4.3068 & 0 & -0.4276 & 0 \\
    0 & 4.3068 & 0 & 0.4276 \\
    -0.0898 & 0 & 16.8427 & 0 \\
    0 & 0.0898 & 0 & 16.8427 \\
    4.3657 & 0 & -13.7123 & 0 \\
    0 & 4.3657 & 0 & 13.7123 \\
    -0.4276 & 0 & 35.1644 & 0 \\
    0 & 0.4276 & 0 & 35.1644
}$$