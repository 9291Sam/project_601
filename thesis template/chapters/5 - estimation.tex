\chapter{Estimation}

For feedback control to be implemented, one needs to have a measurement of the
entire state, $\state$, of a system. When working with a purely mathematical
simulation, this is trivial as one can simply access the desired state and use
it. However, in practical application, the full state is often not directly
available. Sensors can only sense some states therefore some type of estimation
is needed to go from the possible to measure states to the impossible to 
measure ones.

\section{Observability of the ballbot}
For example, without external tracking, there is no sensor that can measure
$x$, $y$ without an external reference. $\dot{x}$ or $\dot{y}$ can be
approximated via the use of optical sensors such as those in the bottom of a 
computer mouse. $\theta$, $\phi$, $\dot{\theta}$ and $\dot{\phi}$ are the most
measurable states because they can be measured with a gyroscope. One should note
here that $\ddot{x}$ and $\ddot{y}$ can be measured by the use of an
accelerometer, but those are not in the state. While this sounds like an
insurmountable problem, in physical application, knowing the precise $x$ and $y$
position isn't actually required, however the other ones are.


The observability of ballbot is entirely dependent on the states that are 
chosen as outputs (the $C$ matrix) or, in robotics terms, which states can be
measured via onboard sensors. This can be done by finding the rank of the 
observability matrix defined $$\mathcal{O} = \begin{bmatrix}
    C \\
    CA \\
    CA^2 \\
    \vdots \\
    CA^{n-1}
\end{bmatrix}$$

\paragraph{Case 1: Position states $x$ and $y$} 
Defining $C = \bmat{1&0&0&0&0&0&0&0\\0&1&0&0&0&0&0&0}$ we can solve for the
observability matrix and find it to be 8, implying that the system is fully
observable. Despite this, I was unable to get a simulation working with this
restricted set of outputs.

\paragraph{Case 2: Gyroscope measureable states $\theta$, $\phi$,
$\dot{\theta}$ and $\dot{\phi}$} 
By following the same method as above, defining 
$C = \bmat{0&0&1&0&0&0&0&0\\0&0&0&1&0&0&0&0\\0&0&0&0&0&0&1&0\\0&0&0&0&0&0&0&1}$
we can find the $rank(\mathcal{O})$ and determine it to be 6. Upon inspection 
of the resultant matrix, one will find that only the $x$ and $y$ position
states to be not observable, this combination are the states that one would
select as the output for the system if one were creating a robot that needed 
to move around

\paragraph{Case 3: Hybrid states $x$, $y$, $\dot{\theta}$ and $\dot{\phi}$}
For the purposes of this project, we have chosen the output matrix to be
$C = \bmat{1&0&0&0&0&0&0&0\\0&1&0&0&0&0&0&0\\0&0&0&0&0&0&1&0\\0&0&0&0&0&0&0&1}$.
This creates a balance where the system is both fully observable, is actually
able to be simulated, and makes the observer fulfil a purpose as there are some 
states that can only be estimated through the use of the observer. Since we
do need to know the precise x and y positions for our desired figure 8 
simulation

\section{Design of the Observer}

For this document, we will use a standard Luenberger Observer taking the form
\begin{align*}
    \dot{\hat{\state}} &= f(\hat{\state}, u) - L(y - \hat{y}) \\
    \hat{y} &= C\hat{\state}
\end{align*}
where $L$ is the observer's gain. To select the observer's gain, we will use
lqr. Selecting the gains was done by  trial and error and eventually resulted in
state gains of $Q = 10 \cdot \text{diag}(1, 1, 25, 25, 1, 1, 5, 5)$ and input
gains of $R = I_{4x4}$.

After solving for this, the resultant gain matrix is
$$
L = \bmat{
    4.3068 & 0 & -0.4276 & 0 \\
    0 & 4.3068 & 0 & 0.4276 \\
    -0.0898 & 0 & 16.8427 & 0 \\
    0 & 0.0898 & 0 & 16.8427 \\
    4.3657 & 0 & -13.7123 & 0 \\
    0 & 4.3657 & 0 & 13.7123 \\
    -0.4276 & 0 & 35.1644 & 0 \\
    0 & 0.4276 & 0 & 35.1644
}$$