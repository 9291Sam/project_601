\chapter{Modeling}
Although a ballbot is a fairly complex system, it can be thought of in various 
ways. The most useful of which, will be as a inverted pendulum affixed atop a 
cart. It's states can be seen from inspection of its movement, while state
equations can be derived by using lagrangian mechanics to derive a continuous
nonlinear time-invariant system.

Upon inspection we can note the following states, additionally shown in \ref{fig:unicycle_states}.
\begin{itemize}
   \vspace{-0.3cm}\item $x$ positional states: $x$ and $\dot{x}$
   \vspace{-0.3cm}\item $y$ positional states: $y$ and $\dot{y}$
   \vspace{-0.3cm}\item Pitch states: $\theta$ and $\dot{\theta}$
   \vspace{-0.3cm}\item Roll states: $\phi$ and $\dot{\phi}$
\end{itemize}

Additionally, we define the following constants that parametrize the simulation.
\begin{itemize}
   \vspace{-0.3cm}\item Radius of the ball: $r$
   \vspace{-0.3cm}\item Distance between center of mass of the ball ball and the center of mass of the body: $d$
   \vspace{-0.3cm}\item Mass of the ball: $m_{ball}$
   \vspace{-0.3cm}\item Mass of the body: $m_{body}$
   \vspace{-0.3cm}\item Moment of the ball around all axes: $I_{ball}$
   \vspace{-0.3cm}\item Moment of the body around the x and y axes: $I_{bodyxy}$
   \vspace{-0.3cm}\item Moment of the body around the z axis: $I_{bodyz}$
   \vspace{-0.3cm}\item Force of gravity: $g$
\end{itemize}

Finally, we define the following inputs to the system.
\begin{itemize}
   \vspace{-0.3cm}\item Ball x Torque: $\tau_x$
   \vspace{-0.3cm}\item Ball y Torque: $\tau_y$
\end{itemize}

Note, that in order to simplify the modeling, we use perpendicular torques
acting on the ball, in practical application, three linearly dependent torques
are actually used. These are equivalent however their relation is not explicitly
defined by this document.

\section{Definition of States}
\begin{figure}[h!]
  \centering
  \includegraphics[width=0.403125\linewidth]{with states.png}
  \caption{Drawing of a Unicycle with states}
  \label{fig:unicycle_states}
\end{figure}

These states, are grouped into a set of generalized coordinates,
$q = [x, y, \theta, \phi]\top$ and then combined to form the full state of the
system.

$$ \state = 
  \bmat{q \\ \dot{q}} =
  \begin{bmatrix}
    x            \\
    y            \\
    \theta       \\
    \phi         \\
    \dot{x}      \\
    \dot{y}      \\
    \dot{\theta} \\
    \dot{\phi}   \\
  \end{bmatrix}
$$

The inputs are represented as a vector $u = \bmat{\tau_x \\ \tau_y}$

\section{Derivation of State Derivatives}

A general nonlinear time-invariant system is defined as
$\dot{\state} = f(\state, u)$, however for the purposes of this document, it is
more helpful to think in the form
$$\dot{\state} = \bmat{\dot{q} \\ \ddot{q}} = f(q, \dot{q})$$

Upon inspection, one can see that both sides of the equation share the state 
$\dot{q}$, showing that one just needs to find a function
$\ddot{q} = f(q, \dot{q}, u)$ which models the dynamics of the system.

The derivation of thus function will be achieved through the use of lagrangian
mechanics. Lagrangian mechanics constructs a function, called the lagrangian
defined as the difference between the kinetic energies $T$ and potential 
energies $U$ of a system:
$$ \mathcal{L} = T - U$$

For the ballbot system, the total kinetic energy is the sum of the translational
and rotational energy of both the ball and the body. Meanwhile, the total
potential energy is defined as the sum of the gravitational potential energy of 
the ball and the body. 

By combining this, we can state the full lagrangian as:
$\mathcal{L} = 
  (E_{ball, translation} + E_{ball, angular} + E_{body, translation} + E_{body, angular})-
  (E_{ball, height} + E_{body, height})
$

\subsection{Derivation of translational energies} \label{der_translation}

The translational kinetic energy of any rigid body is defined by the velocity
of its center of mass. While the scalar form $E = \frac{1}{2} m * v^2$ is 
familiar, in the context of this document, it will need to be extended into the
vector case. This more general equation is written:
$$E = \frac{1}{2} m *  v^\top * v$$

In our case, defining these velocities is tricky, so we rely on the fact that
the time derivative of the position vector is the velocity vector. This final
equation is written:
$$E = \frac{1}{2} m *  \dot{p}^\top * \dot{p}$$

where $p$ is the position vector in the frame of the object. This general
formulation allows us to derive the translational energy of an object from it's
position vector.

\paragraph{Translational energy of the ball}

Starting off with the position of the ball, it can be easily seen that the
position of the center of mass of the ball will be the sum of the position of
it's base plus the radius of the ball upwards. 
$$p_{ball} = \bmat{x \\ y \\ 0} + \bmat{x \\ y \\ r} = \bmat{x \\ y \\ r}$$

Taking the time derivative of this vector yields 
$$v_{ball} = \dot{p}_{ball} = \bmat{\dot{x} \\ \dot{y} \\ 0}$$

Finally, this can be inputted into the translational energy formula to arrive at
$$E_{ball, translation} = \frac{1}{2} m *  v_{ball}^\top * v_{ball}$$

\paragraph{Translational energy of the body}

The position of the body's center of mass is more complex as it depends not only
on the position of the ball, but also on the angles of the body ($\theta$ and
$\phi$) and the distance between the ball's center of mass and the body's center
of mass ($d$)

However, the rotation of the body is defined by the rotation matrix sequence
of rotating around the X axis an angle $\phi$ followed by a rotation around
the Y axis an angle $\theta$. These orations can be codified in the rotation 
matrices:
$$
R_{body} = R_y(\theta) R_x(\phi) = 
\bmat{
\cos\theta & 0 & \sin\theta \\ 
0 & 1 & 0 \\ 
-\sin\theta & 0 & \cos\theta 
}
\bmat{ 
1 & 0 & 0 \\ 
0 & \cos\phi & -\sin\phi \\ 
0 & \sin\phi & \cos\phi 
}
$$

By using these matrices, one can easily calculate the position of the center
of mass of the body as
$$p_{body} = p_{ball} + R_{body} \bmat{ 0 \\ 0 \\ d }$$

Therefore, using the same method as established previously, one can calculate
the velocity as the time derivative of the position vector. Do note that, 
because of the chain rule, and the resulting expression being dependent on 
$\theta$ and $\phi$ that the jacobian should be used in the form
$$v_{body} = \frac{d p_{body}}{dt} = \frac{\partial p_{body}}{\partial q} \frac{\partial q}{\partial t}$$

Resulting in a similar relation of
$$E_{body, translation} = \frac{1}{2} m *  v_{body}^\top * v_{body}$$

\subsection{Derivation of angular energies}

In a similar method as in \ref{der_translation}, one can start with the equation
for the angular energy and similarly extend it to multiple. The scalar form is
$E = \frac{1}{2} I \omega^2$ where $I$ is the moment of inertia of the spinning
body and $\omega$ is the angular velocity of that body. In the vector case,
this similarly transforms into 
$$E = \frac{1}{2} \omega^\top * I * \omega$$. 

\paragraph{Rotational energy of the ball}

Since the ball never slips on the ground, it's angular velocity is directly
derived from the linear velocity of the ball. For a ball of radius $r$, motion
in the +X axis, induces a rotation around the +y axis. And, similarly, 
motion in the +y axis induces a rotation round the -x axis because of the right
handed axes being used. Therefore,

$$\omega_{ball} = \bmat{-\dot{y}/r \\ \dot{x}/r \\ 0}$$

\paragraph{}



Calculating these linear accelerations is possible by splitting the linear 
distance function into its components via $\theta$ and then repeatedly deriving
with respect to linear Additionally, on inspection we can see that angular
acceleration will be linearly related to linear acceleration and the
wheel's radius.

Starting with the equation for the arc length of a wheel, $s = r\Omega$ where
$s$ is the distance swept by a wheel of radius $r$ over an angle $\Omega$ (note
the intentional reuse of $\Omega$). This equation can be derived, into the
formula for the linear velocity of processing wheel $v = r\omega$ where $v$ is
the velocity of the traveling wheel in the direction of motion and $\omega$ and
$\dot{\Omega}$ are the angular velocities of the wheel. By splitting this
equation into two based on components we arrive at
\begin{align*}
  v_x = \dot{x} &= r \cdot \omega \cdot \cos(\theta) \\
  v_y = \dot{y} &= r \cdot \omega \cdot \sin(\theta) 
\end{align*}

Finally, the derivative of this set of equations can be taken one more time find
the desired formula for linear acceleration.
\begin{align*}
  \ddot{x} &= r (\dot{\omega} \cos(\theta) - \omega \sin(\theta)\dot{\theta}) \\
  \ddot{y} &= r (\dot{\omega} \sin(\theta) + \omega \cos(\theta)\dot{\theta})
\end{align*}

\paragraph{Rotation $\theta$}
Upon inspection, one notices that $\theta$ only changes based off of $\phi$. In
order for a unicycle to change it's $\theta$, the rider must lean to one side
and change $phi$ first!

Imagine a unicycle balanced slightly over at an angle and going forward in its
own local frame of reference. As it unicycle moves forward, it will rotate
through $\theta$ tracing a perfect arc, dependent on the body angle $\phi$. By
trying some values, it can be noticed that if an angle $\phi = 0$ the radius
of the circle traces goes to infinity. Meanwhile, if the angle is larger, 
something closer to $\phi = \pm \frac{\pi}{8}$, then theta will slowly change
based on the current phi. 

With this in mind, the formula for this arc is the turning radius ($r_t$) of a
tilted. $r_t = \frac{r}{\sin(\phi)}$, the derivation of which is shown in
\ref{fig:turn_radius_derivation}, can be used to relate $r_t$, $r$, and $\phi$.
However, we now need a relation between $r_t$ and $\dot{\theta}$. This can be
derived from the formula for arc length used previously. The formula
$s = r\theta$, can once again be reinterpreted as $s = r_t\theta$ where $s$ the
arc length of the turn radius that is swept by a wheel of radius $r$ traveling=
over that arc an angle $\theta$. This means we can take the derivative of this
formula to arrive at  $v = r_t\dot{theta}$. Finally, since we don't store the
wheel's local forward velocity we can rewrite it in terms of angular
acceleration as $\dot{\theta} = \frac{r\omega}{r_t}$. Now, by plugging in the
formula derived earlier, we finally achieve
$\dot{\theta} = \frac{r\omega\sin(\phi)}{r} = \omega\sin(\phi)$.
Finally, by applying the product rule we can find the desired formula for
$\ddot{\theta}$

$$\ddot{\theta} = \dot{\omega}\sin(\phi) + \omega \dot{\phi}\cos(\phi)$$

\begin{figure}[h!]
  \centering
  \includegraphics[width=0.525\linewidth]{derivation_turn_radius.png}
  \caption{Drawing of a Unicycle with states}
  \label{fig:turn_radius_derivation}
\end{figure}

\paragraph{Rotation $\phi$}
Upon inspection, one notices that if perfectly balanced, a unicycle stands
perfectly vertical it will stay there, however any slight perturbation will
cause it to fall over. Additionally, the only perturbation that could cause this
is the torque $\tau_2$ as every other input acts such that no matter its value,
there will be no change on $phi$. 

From this, assuming the unicycle is not moving forward, we can create a 
simplified model of it. This is model in REFERENCE FIGURE. On this model, we 
have the force of gravity acting at both the center of mass of the wheel and 
on the rider. Additionally, we have a torque acting on the rider's center of
mass. The gravitational forces can be split up into components along and
perpendicular to the unicycle, with those along being canceled by the normal
force of the unicycle. This leaves $F_{gravity, wheel} \sin(\phi)$, 
$F_{gravity, rider} \sin(\phi)$ and $\tau_2$. By using newton's third law for
rotation we see that the sum of torques $\sum \tau = I\alpha$ or, restated into
the conventions of this document, $\sum = I \ddot{\phi}$ must be related. By
combining all of these, we arrive at the formula
$$I_{pivot} \ddot{\phi} = m_wrg\sin(\phi) + m_rg(r+d)\sin(\phi)+\tau_2$$
where $I_{pivot} = m_wr + m_r(r+d)$ and is the moment corresponding to the pivot
point being at the point of contact of the wheel.

This formula is accurate in the case of the unicycle when there is zero motion
however once the unicycle starts to move, there are additional forces in play,
the centrifugal force and the gyroscopic force. 

Since we are working with torques, we need the formula for the torque caused
by the centrifugal force. This can be interpreted as a torque occurring at the
center of mass of the combined system of the rider and wheel perpendicular to
the body of the unicycle, that is $\tau_c = d_cF_cm_c\cos(\phi)$ where
$d_c$ is the distance from the pivot to the center of mass, $F_c$ is the
centrifugal force, $m_c$ is the total mass of the system. Additionally, note the
$\cos(\phi)$ term, because the centrifugal force acts parallel to the global y
axis regardless of the angle $\phi$ of the unicycle, it must be broken down into
components that are cancelled by the unicycle's normal force and those which act
as a net torque perpendicular to the axis of travel. 

Starting off with $d_c$, it can be found by treating the unicycle as a seesaw
and finding the location at which the torques balance. This formula is defined
in \ref{par:d_c derivation} and resolves to
$d_c = \frac{m_rr + dm_r + m_wr}{m_w+m_r}$. Next, is
$F_c$, which additional is defined in \ref{par:F_c derivation} to be
$F_c = r_w \omega \dot{\theta} m_c$.
These three formulas can be combined together to form
$$ \tau_c = r_w \omega \dot{\theta} m_c d_c \cos(\phi)$$

Finally, the gyroscopic torque can be calculated and, as detailed in
\ref{par:t_g derivation}, its the formula is
$\tau_g = I_{wz} \omega \dot{\theta} \cos(\phi)$

This can now be stated as 
$$\ddot{\phi} = \frac{\tau_{gravity} + \tau_2 + \tau_c + tau_{gyroscopic}}{I_{pivot}}$$

\paragraph{Rotation $\omega$ and $\alpha$} 
The omega and alpha terms are special as they are interlinked. We will address
their derivations separately, resolving two separate equations, both dependent 
on $\dot{\omega}$ and $\ddot{\alpha}$ and then use algebraic substitution. 
Their derivations are explained in detail in the Appendix; however, here we 
will place their full definitions, where the system determinant is defined as
$\Delta = [ (m_w + m_r)r^2 + I_{wy} ] [ I_{wy} + m_r d^2 ] - ( m_r r d \cos \alpha )^2$.

The explicit formula for wheel acceleration is:
$$
\dot{\omega} = \frac{ ( I_{wy} + m_r d^2 ) (m_r g d \sin \alpha) - (m_r r d \cos \alpha)(m_r g d \sin \alpha - \tau_1) }{ \Delta }
$$

The explicit formula for pitch acceleration is:
$$
\ddot{\alpha} = \frac{ [ (m_w + m_r)r^2 + I_{wy} ] (m_r g d \sin \alpha - \tau_1) - (m_r r d \cos \alpha)(m_r g d \sin \alpha) }{ \Delta }
$$

\paragraph{Final state equation}
Now that each sub-equation has been derived and the dependencies substituted,
the final non-linear state update vector $\dot{x}$ can be written.

$$
\dot{x} = \begin{bmatrix}
  \dot{x} \\
  \dot{y} \\
  \dot{\theta} \\
  \dot{\phi} \\
  \omega \\
  \dot{\alpha} \\
  r (\dot{\omega} \cos\theta - \omega \dot{\theta} \sin\theta) \\
  r (\dot{\omega} \sin\theta + \omega \dot{\theta} \cos\theta) \\
  \dot{\omega}\sin\phi + \omega\dot{\phi}\cos\phi \\
  \dfrac{ [ m_w r + m_r (r+d) ] (g\sin\phi + r \omega \dot{\theta} \cos\phi) - I_{wz} \omega \dot{\theta} \cos\phi + \tau_2 }
    { m_w r^2 + m_r(r+d)^2 + I_{wy} } \\
  \dfrac{ ( m_r d^2 ) (\tau_1) - (m_r r d \cos \alpha)(m_r g d \sin \alpha - \tau_1 + \tau_3) }
    { [ (m_w + m_r)r^2 + I_{wy} ] [ m_r d^2 ] - ( m_r r d \cos \alpha )^2 } \\
  \dfrac{ [ (m_w + m_r)r^2 + I_{wy} ] (m_r g d \sin \alpha - \tau_1 + \tau_3) - (m_r r d \cos \alpha)(\tau_1) }
    { [ (m_w + m_r)r^2 + I_{wy} ] [ m_r d^2 ] - ( m_r r d \cos \alpha )^2 }
\end{bmatrix}
$$


Some interesting things can be gleamed from this equation. Most notably the
fact that $\tau_1$ and $\tau_3$ are highly coupled but not identical!