\chapter{Ballbot Simulation}

The following sections visually demonstrate the uncontrolled dynamics of the
ballbot under various initial conditions.


\section{No Deviation}
The following are graphs showing the ballbot stand perfectly still at its
unstable equilibrium point, $\state = 0$. Both graphs are as expected as
although the equilibrium point is unstable, no deviations are made to cause the
ballbot to move from it. 

\begin{figure}[h!]
  \centering
  \includegraphics[width=0.96\linewidth]{uncontrolled_xtheta_no_deviation.png}
  \caption{Graph of the $x$ and $\theta$ coordinates of the ballbot at its equilibrium point}
  \label{fig:uncontrolled_xtheta_no_deviation}
\end{figure}


\section{Theta Deviation}

Figures \ref{fig:uncontrolled_x_theta_deviation} and
\ref{fig:uncontrolled_theta_theta_deviation} represent the ballbot starting with 
$\state = \bmat{0&0&0.00001&0&0&0&0&0}^\top$. These behave mostly as expected,
with two things of note. First, is the slight positive $x$ drive in 
\ref{fig:uncontrolled_x_theta_deviation}, we assume this to be the result of
simulation inaccuracy. Second, is the increasing angle of theta, which is odd
but not a problem as we are not attempting to control the system. A more
rigorous simulation would have something in place to prevent this. 

\begin{figure}[h!]
  \centering
  \includegraphics[width=0.73125\linewidth]{uncontrolled_x_theta_deviation.png}
  \caption{Graph of the $x$ coordinate of the ballbot starting near its equilibrium state with a slight deviation to $\theta$}
  \label{fig:uncontrolled_x_theta_deviation}
\end{figure}

\begin{figure}[h!]
  \centering
  \includegraphics[width=0.73125\linewidth]{uncontrolled_theta_theta_deviation.png}
  \caption{Graph of the $\theta$ coordinate of the ballbot starting near its equilibrium state with a slight deviation to $\theta$}
  \label{fig:uncontrolled_theta_theta_deviation}
\end{figure}


\section{Theta and Phi Deviation}

Figures \ref{fig:uncontrolled_x_both_deviation}, 
\ref{fig:uncontrolled_y_both_deviation}, 
\ref{fig:uncontrolled_theta_both_deviation}, and
\ref{fig:uncontrolled_dottheta_no_deviation} start with 
$\state = \bmat{0&0&0.00001&0.0001&0&0&0&0}^\top$. At these initial conditions
we can start to notice sone chaotic behavior arising from the interaction
between the nonlinear pendulum force and the linear kickback of the wheel from
the falling body.

\begin{figure}[h!]
  \centering
  \includegraphics[width=0.74\linewidth]{uncontrolled_x_both_deviation.png}
  \caption{Graph of the $x$ coordinate of the ballbot starting near its equilibrium state with a slight deviation to $\theta$ and $\phi$}
  \label{fig:uncontrolled_x_both_deviation}
\end{figure}

\begin{figure}[h!]
  \centering
  \includegraphics[width=0.74\linewidth]{uncontrolled_y_both_deviation.png}
  \caption{Graph of the $y$ coordinate of the ballbot starting near its equilibrium state with a slight deviation to $\theta$ and $\phi$}
  \label{fig:uncontrolled_y_both_deviation}
\end{figure}

\begin{figure}[h!]
  \centering
  \includegraphics[width=0.74\linewidth]{uncontrolled_theta_both_deviation.png}
  \caption{Graph of the $\theta$ coordinate of the ballbot starting near its equilibrium state with a slight deviation to $\theta$ and $\phi$}
  \label{fig:uncontrolled_theta_both_deviation}
\end{figure}

\begin{figure}[h!]
  \centering
  \includegraphics[width=0.74\linewidth]{uncontrolled_dottheta_no_deviation.png}
  \caption{Graph of the $\dot{\theta}$ coordinate of the ballbot starting near its equilibrium state with a slight deviation to $\theta$ and $\phi$}
  \label{fig:uncontrolled_dottheta_no_deviation}
\end{figure}