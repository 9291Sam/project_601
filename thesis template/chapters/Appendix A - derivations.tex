\chapter{Appendix A - Derivations}

\paragraph{Derivation of $d_c$ for $\phi$} \label{par:d_c derivation}

Looking at Figure \ref{fig:derivation_com}, $x$ is the distance from the base to
the center of mass, the variables in the drawing perfectly correspond to those
used in the derivation of $\phi$. By balancing moments, with respect to x, we
arrive at the equation $(x-r)m_w = (r+d-x)m_r$ and, by solving for x, we arrive
at
$$ x = \frac{m_rr + dm_r + m_wr}{m_w+m_r}$$

\begin{figure}[h!]
  \centering
  \includegraphics[width=0.525\linewidth]{center_of_mass.png}
  \caption{Drawing of a Unicycle's center of mass points}
  \label{fig:derivation_com}
\end{figure}



\paragraph{Deivation of $F_c$ for $\phi$} \label{par:F_c derivation}

As a reminder, $F_c$ is the centrifugal force acting upon the entire unicycle
at the center of mass \textbf{parallel to the global y axis}. to the frame of
the unicycle. Starting off with Newton's third law, $F = ma$, we see that in
order to calculate this, we need the total mass of the system, $m_c = m_w + m_r$
and the centrifugal acceleration. The formula for centrifugal acceleration is 
$a_c = \frac{v^2}{r}$ where v is the velocity in the direction of travel and $r$
is the radius swept by the rotation. Thankfully, both of these correspond to
formulas previously derived. $v=r_t\dot{\theta}$ restated as
$r_t = \frac{v}{\dot{\theta}}$ and $v = r_w\omega$.
Combing these together we arrive at
$a_c = \frac{v^2}{v / \dot{\theta}} = v\dot{\theta} = r_w\omega\dot{\theta}$.
Finally, this can be substituted into Newton's third law to achieve
$$F_c = r_w\omega m_c \dot{\theta}$$

\paragraph{Derivation of gyroscopic effect} \label{par:t_g derivation}

The angular momentum of a spinning wheel is $L = I_w*\omega$ keeping with the 
conventions of this document, that means $L = I_{wy}\omega$. Additionally, 
newton's second law for rotation states that $\tau = \dot{L}$ By the small
angle approximation, we can approximate $\dot{L} = L\dot{\theta}$ and with this
we arrive at the formula $\tau_g = I_w\omega\dot{\theta}$. Finally, remember
that this torque must act opposite as a restoring force, as it is currently
written as a positive feedback loop.
$$\tau_g = -I_w\omega\dot{\theta}$$