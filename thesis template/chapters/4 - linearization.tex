\chapter{Linearization}

Linearization is the method of taking a time-invariant continuous-time system
$\dot{\state} = f(\state, u)$ and approximating it as a linear time-invariant
system of the form 
$$\dot{\state} = A\state + Bu$$

A linearization is calculated by evaluating the jacobian of the function $f$
with respect to the state or the input. Put symbolically, given an equilibrium
state $\tilde{\state}$ and an equilibrium input $\tilde{u}$,
$$A = \frac{\partial f}{\partial \state}(\tilde{\state}, \tilde{u})$$
and
$$B = \frac{\partial f}{\partial u}(\tilde{\state}, \tilde{u})$$

For the ballbot, this means calculating it's equilibrium points and the jacobian
of the full state update function.

\section{Equilibrium points of the ballbot}

Equilibrium points are defined as states where the system dynamics are
stationary, implying $$\dot{\state} = f(\tilde{\state}, u) = 0$$

From the definition of $f$ as calculated before, one can see that any
equilibria point of the ballbot will be dependent on the state and the input.
Looking first toward the state $\state = \bmat{x&y&\theta&\phi&\dot{x}&\dot{y}&\dot{\theta}&\dot{\phi}}^\top$

In order for $\dot{\state} = 0$ to hold true the state must have its angles
($\theta$ and $\phi$), translational velocities ($\dot{x}$ and $\dot{y}$), and
angular velocities ($\dot{\theta}$ and $\dot{\phi}$) all be equal to zero. 
Interestingly, this means that the ballbot can be linearized around any vertical
position, as all are reasonable.

Moving onto the inputs, it's trivial to see that in order for $\dot{\state} = 0$
to hold, $u = 0$ must be true. As any input would immediately knock the system
out of the equilibria point

\section{Resultant A and B matrices}

Using the above equations, one can pick a value for $\tilde{\state}$ and use
it to calculate the $A$ and $B$ matrices. Doing this symbolically in Matlab,
when combined with the derivation steps as derived earlier, results in the
following matrices. Note the inclusion of the shared denominator defined
$ D = I_{ball}I_{xy} + I_{ball}m_{body}d^2 + I_{xy}m_{ball}r^2 + I_{xy}m_{body}r^2
+ m_{ball}m_{body}r^2d^2 $

$$
A = \bmat{
  0 & 0 & 0 & 0 & 1 & 0 & 0 & 0 \\
  0 & 0 & 0 & 0 & 0 & 1 & 0 & 0 \\
  0 & 0 & 0 & 0 & 0 & 0 & 1 & 0 \\
  0 & 0 & 0 & 0 & 0 & 0 & 0 & 1 \\
  0 & 0 & -\frac{m_{body}^2 g r^2 d^2}{\Delta} & 0 & 0 & 0 & 0 & 0 \\
  0 & 0 & 0 & \frac{m_{body}^2 g r^2 d^2}{\Delta} & 0 & 0 & 0 & 0 \\
  0 & 0 & \frac{m_{body} g d (I_{ball} + m_{ball}r^2 + m_{body}r^2)}{\Delta} & 0 & 0 & 0 & 0 & 0 \\
  0 & 0 & 0 & \frac{m_{body} g d (I_{ball} + m_{ball}r^2 + m_{body}r^2)}{\Delta} & 0 & 0 & 0 & 0
}
$$

$$
B = \bmat{
  0 & 0 \\
  0 & 0 \\
  0 & 0 \\
  0 & 0 \\
  0 & -\frac{r(I_{xy} + m_{body}d^2 + m_{body}rd)}{\Delta} \\
  \frac{r(I_{xy} + m_{body}d^2 + m_{body}rd)}{\Delta} & 0 \\
  0 & \frac{I_{ball} + m_{ball}r^2 + m_{body}r^2 + m_{body}rd}{\Delta} \\
  \frac{I_{ball} + m_{ball}r^2 + m_{body}r^2 + m_{body}rd}{\Delta} & 0
}
$$

As expected from the equilibria derivation, the first two columns of the
linearization are zeros implying that the linearization is independent of
the state's $x$ and $y$ values. Additionally, note the one to one mapping
of the $\dot{q}$ vectors in the upper right corner of the matrix.